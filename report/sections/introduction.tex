\section{Introduction and Background}
  	%% Main Aim
		

		%% Relevant materials * 3 - Some materials also on real world ant colonies
		% The below are a list of references that are not in the .bib file yet, will be added when I'm (Blayze) back home
		% 1) https://cs.gmu.edu/~sean/papers/panait04ant.pdf
		% 2) https://ieeexplore.ieee.org/document/4129846/ - ACO algorithm
		% 3) https://ieeexplore.ieee.org/document/4375636/ - ACO again
		% 4) https://link.springer.com/chapter/10.1007%2F978-3-540-28646-2_17 - Real ant simulation
		% 5) https://cs.gmu.edu/~sean/papers/panait04learning.pdf --- TODO: Read this some more, read the referenced papers, make sure it's being referenced properly

		A number of past studies have examined the way that ant colonies search their surrounding areas to locate and retrieve food \cite{vittori_modeling_2004, 1}.
		Often the aim of those studies focused on simulating ant colonies is to apply the emergent behaviour that ant colonies exhibit to solve abstracted technical
		problems \cite{2, 3}, with a smaller number of past works focusing on simulating ant colony behaviour as it is seen in nature \cite{4} %TODO: Perhaps cite some more things here
		
		\cite{4} paper implements ant trail following  ---- somehting about it modelling using deposition
		
		Whereas this \cite{1} introduces the concept of a dual-pheromone system, where one set of pheromone leads away, one toward
		
		\cite{5} discusses the benefits of simulating the ant's return path using pheromones or simply having the ants return directly

		TODO: Cite one other thing here, giving it another piece of information

		%% How the relevant materials inspired us
		These pieces of information inspired us to do stuff like the other stuff
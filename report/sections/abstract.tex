\begin{abstract}
 
  %\section*{Abstract}
    %Background
    Ant colonies have the unique ability to dynamically search the terrain surrounding their environments, detect and retrieve sources of food. This is an emergent behaviour that
    has been well well explored by a variety of past studies \cite{camazine_self-organization_2003} %TODO: Add more citations here.
    often with a focus on examining the way that paths to and from food sources are created and organised. %TODO: More reference to this particular thing here.
    In reality, many species of ants have a number of different castes of ants within their ranks \cite{noauthor_caste_nodate}. This study examines what affect different distributions
    of two castes of ants in particular -- the \textit{minor-} and \textit{major-worker} ants -- within a colony have on that colony's overall health when faced with differing conditions.\par
    %Methods                                                                      v TODO here? Discuss specifics of descretisation (size, sector size)
    Ant colonies are simulated in Matlab on train that has been descritised into chunks, and food is spawned randomoly throughout the terrain. Ant colonies with differing ratios of simulated
    minor and major worker ants are created, and their success -- as measured by [MEASUREMENT HERE].%TODO: What are we measuring for colony health? Population size? Total energy gathered? Average Energy over a timeframe?
    %Results
    We found that [FINDINGS HERE]
    %Conclusions                                                          v TODO: At this point, I'm just guessing
    These findings indicate that [INSIGHT INTO ANTS HERE], indicating that a mix of major and minor workers is the most suitable -- with the minor workers excelling at [SOMETHING?] while the
    major workers [DO SOMETHING ELSE], as seen in nature [CITE THIS ACTUALLY HAPPENING - Or, I guess don't if it doesn't really happen and just state that our model was wrong.]
  \end{abstract}

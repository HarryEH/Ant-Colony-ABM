\begin{abstract}
    %Background
    Ant colonies have the unique ability to gather food from their surrounding environments using dynamic, emergent behaviours that 
    have been well explored by a variety of past studies \cite{camazine_self-organization_2003, Burd2003, Robinson2008, HERBERS1996141, vittori_modeling_2004, a_panait_ant_2004},
    often focusing on examining the paths formed to and from food sources \cite{a_panait_ant_2004, a_panait_learning_2018, vittori_modeling_2004}.
    Many species of ants have a number of different castes within their ranks \cite{noauthor_caste_nodate}. This study examines what affect different distributions
    of two castes of ants in particular -- the \textit{minor} and \textit{major} -- worker ants within a colony have its overall ability gather food.

    %Methods
    Ant colonies with descritised terrain and differing ratios of worker-caste ants are  simulated and food is spawned randomly throughout. 
    Their success -- the total energy collected -- is plotted against these ratios in order 
    to identify the most sucessfull distribution of minor and major workers.

    %Results
    Our model seems to emulate the behaviour of ant colonies in the real world, with the model indicating that a ratio of around 32\% minor to 68\% major workers is the most
    successful, reflecting the average distribution of workers seen in mature nests in the real-world as measured in a previously published work \cite{Tschinkel1988}.
\end{abstract}

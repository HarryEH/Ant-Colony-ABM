\begin{abstract}
    %Background
    Ant colonies have the unique ability to dynamically search the terrain surrounding their environments, detect and retrieve sources of food. This is an emergent behaviour that
    has been well explored by a variety of past studies \cite{camazine_self-organization_2003, Burd2003, Robinson2008, HERBERS1996141, vittori_modeling_2004, a_panait_ant_2004}
    often with a focus on examining the way that paths to and from food sources are created and organised\cite{a_panait_ant_2004, a_panait_learning_2018, vittori_modeling_2004}.
    In reality, many species of ants have a number of different castes of ants within their ranks \cite{noauthor_caste_nodate}. This study examines what affect different distributions
    of two castes of ants in particular -- the \textit{minor} and \textit{major} -- worker ants within a colony have on that colony's overall health when faced with differing conditions.

    %Methods
    An ant colony is simulated in Matlab on terrain that has been descritised into chunks, and food is spawned randomoly throughout the terrain. Ant colonies with differing ratios of simulated
    minor and major worker ants are created, and their success -- as measured by the total energy collected from spawned food sources -- is plotted against these ratios in order to attempt
    to identify the most sucessful distribution of minor and major worker ants.

    %Results
    We found that our model seemed to emulate the behaviour of ant colonies in the real world, with the model indicating that a ratio of around 35\% minor to 65\% major worker ants is the most
    successful, reflecting the average distribution of minor to major worker ants seen in real-world mature nests as mesured in a previously published work\cite{Tschinkel1988}.
\end{abstract}

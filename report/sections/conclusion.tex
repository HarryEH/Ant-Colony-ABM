 \section{Conclusions}
  % Here the results should be compared with the distributions of minor and major workers seen in real-world colonies.
  % It's important to have a section here talking about how we have a simulation and not a real colony.
In conclusion, the main take-away from this report and the results of our simulated ant colony is that, like in the real world, a mix of major and minor workers is optimal for the overall `energy' of the colony. For our system a ratio of 32:68 major to minor worker ants was optimal based on the seeds that we used. In a 1988 journal article \cite{Tschinkel1988} Walter Tschinkel, a renowned myrmecologist, gave the ratios in a mature fire ant colony as 35:65 of major to minor workers. Given that the optimal value for our colony is close to this, it can be concluded that our model gives a reasonable representation of optimal worker ant ratios for foraging, although in order to add additional weight to our results further simulations and evaluations could still be performed with larger colony sizes and a larger sample size of initial seed conditions.

Overall, the use of an agent based model proved to be an effective way of simulating ant colony behaviour. However, is it important to note that given that our colony is unrealistically small and other assumptions have been made, we should not make too much of this positive result, instead encouraging further work.

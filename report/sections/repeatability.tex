\section{Repeatability}

Given the nature of this simulation it is important that our results are repeatable. This is, obviously, the case for this project. We used `rng(seed)' with a seed value to set the seed used for generating random numbers. For an example of the repeatability, run `simple\_example' exactly as it has been handed in. Although if you have changed parameters, use the values in the table \ref{tab:repeat}. This will give you a value of `6530' displayed in the console and saved to 'final\_energy'. Hence, as you can see we have repeatable results.\par

\begin{table}[htb]
\centering
\caption{This table gives the variable values for the file simple\_example.m. They should be this for the hand-in but they are here for repeatability.}
\label{repeat}
\begin{tabular}{@{}cc@{}}
\toprule
Variable          & Value \\ \midrule
seed              & 2     \\ \midrule
environment\_size  & 50    \\ \midrule
colony\_count      & 1     \\ \midrule
worker\_percentage & 0.5   \\ \midrule
colony\_ants\_total & 30    \\ \midrule
simulation\_length & 1200  \\ \bottomrule
\end{tabular}
\end{table}


To get the graphs that we have produced, we ran the experiments in parallel. If you run `ant\_parallel' you will be able to generate exactly the same graphs. However, if you don't have a quad core processor you may have to change `cpu = 4;' to something that is a better value for you. If you don't want to run in parallel then set this to `cpu = 1'. However, this will will increase the time it takes for the simulations to finish.\par
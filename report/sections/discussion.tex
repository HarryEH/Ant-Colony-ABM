\section{Discussion}

% Refernce https://link.springer.com/article/10.1007/BF00303545 here (adding to refs.bib to be done), comparing the success of the colony
% against the success of examined colonies in that paper at various points in time where the colony is similar to our simulation as current.
% Our system currently says that more major workers are better - this is a change from the stuff recorded in the journal article by Tschinkel (1988) which indicated that only 35% of the workers were majors.
  % This could be because our system does not need to deal with the winter periods, which that same paper shows has a significant effect on the number of major workers, a setback our system never has to deal with.

% List of assumptions we can talk about
% - Whenever we're comparing about the article by Tschinkel, we've gotta remember that his article was based on _fire ants_ not all ants.
% - food decays in the real world (though over a long period of time, so perhaps not as relevant to our own system.)

% Might be worth talking about how our system relates to the work cited in the introduction, particularly a_panait_ant_2004 and `An agent-based model of collective nest choice by the ant Temnothorax albipennis'
\subsection{Implications, Reliability and Realism}
Our model indicates that the optimum ratio for minor to major worker ants is 35:65. This reflects very well the findings that Tschinkel et al. found in their 1988 paper \cite{Tschinkel1988} that we originally examined when searching for a baseline reference of real-world data. In order to ensure that our results provided a true representation of the average behaviour of our model at the various ratios, each ratio
simulation was run 5 separate times and an average of the energy collection was used, limiting the chance of abnormal results being reported. This number of repeats, however, could be higher in order to gain a better sample of possible simulation runs, further increasing the representative power of these experiments with respect to our model's overall behaviour.

Another limitation of the parameters we used when experimenting with the various ratios was the number of ants in each simulation. Each simulation that we ran only contained 30 ants. This limited the the resolution of possible in our distribution experiments, as the minimum unit of change we could make was one ant, which equates to a change of 3.3\% in worker
ratio. This limited our ability to observe fine-grained results when narrowing our experimental range in order to observe the ratio-efficiency distribution around the 35\% mark. This also indicates an area where our model begins to lack realism, as real-world ant colonies can have a significantly larger population (Tschinkel et al. measure 220000 workers in colonies 4 to 6 years old \cite{Tschinkel1988}).

It is possible for our model to run larger colony sizes and an increased number of times, however doing so requires significantly more computation time and would be outside of the scope of this study.

\subsection{Future Improvements}
By improving the computational efficiency of the simulation, tests can be run for longer amounts of time to provide longer term trends and repeated to give averaged results. Implementing the improvement suggested in the computation efficiency section of this report would greatly improve the run-time of the simulation and such improvements to computational efficiency will allow for larger amounts of experimentation.\par
In order to further improve the model itself, there are several properties of real ant colonies which could be modelled to improve biological accuracy. One example of this is reproduction, by giving ant colonies the ability to reproduce would allow the simulation to run over longer periods of time and provide long term trends of ant behaviour. The current system does not include reproduction and so is only suitable for shorter simulations, where reproduction has very little impact. If a colony were to have a large amount of ants die, then as the environment becomes dense with food, the population will slowly increase again until an equilibrium point is reached. Modelling such behaviour would provide results which are more closely aligned to nature, where colonies do have the ability to produce new ants.\par
Another area of improvement could be to simulate colony interactions. In the current system several colonies can be simulated simultaneously, however ants from two colonies do not interact. Giving ants the ability to fight each other would allow for colonies to compete for food sources. If the simulation was extended to include ant interactions and reproduction, the simulation would likely depict a more natural representation of ant colony behaviour.\par

\subsection{Modelling Ant Colonies with an Agent Based Model}
In order to simulate the behaviour of an ant colony, each individual ant was modelled as an agent and functioned following a list of behaviours. This agent based approach allowed for emergent behaviour to result from a collection of relatively simple rules. This being the main advantage of an agent based model, behaviour becomes easy to modify, understand and the simulation of ants can be extended to include new behaviours with ease. However, the primary disadvantage of using an agent based model in this instance was that by modelling each individual ant, the computational costs become very large. When compared with a mathematical model, when considering the type of system developed, it can be considered appropriate that an agent based model was used because the mathematical equation of such a complex system could not easily be understood, tested and modified.  

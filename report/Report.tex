\documentclass{article}

\usepackage{graphicx}
\usepackage{listings}
\usepackage{subfig}
\usepackage{mathtools}
%\usepackage{subfigure}
\graphicspath{ {images/} }

\title{
  Simulating Ant Colonies to Investigate How Woker/Scout distributions Affect Colony Health
  \small Harry Howarth, Frazer Bennett-Wilford, Blayze Millward, Daeun Kim \\
}
\author{Harry Howarth, Frazer Bennett Wilford, Daeun Kim}

\begin{document}
  
  \section*{Abstract}
    %Background
    Ant colonies have the unique ability to dynamically search the terrain surrounding their environments, detect and retrieve sources of food. This is an emergent behaviour that
    has been well well explored by a variety of past studies \cite{camazine_self-organization_2003} %TODO: Add more citations here.
    often with a focus on examining the way that paths to and from food sources are created and organised. %TODO: More reference to this particular thing here.
    In reality, many species of ants have a number of different castes of ants within their ranks \cite{noauthor_caste_nodate}. This study examines what affect different distributions
    of two castes of ants in particular -- the \textit{minor-} and \textit{major-worker} ants -- within a colony have on that colony's overall health when faced with differing conditions.
    %Methods                                                                      v TODO here? Discuss specifics of descretisation (size, sector size)
    Ant colonies are simulated in Matlab on train that has been descritised into chunks, and food is spawned randomoly throughout the terrain. Ant colonies with differing ratios of simulated
    minor and major worker ants are created, and their success -- as measured by [MEASUREMENT HERE].%TODO: What are we measuring for colony health? Population size? Total energy gathered? Average Energy over a timeframe?
    %Results
    We found that [FINDINGS HERE]
    %Conclusions                                                          v TODO: At this point, I'm just guessing
    These findings indicate that [INSIGHT INTO ANTS HERE], indicating that a mix of major and minor workers is the most suitable -- with the minor workers excelling at [SOMETHING?] while the
    major workers [DO SOMETHING ELSE], as seen in nature [CITE THIS ACTUALLY HAPPENING - Or, I guess don't if it doesn't really happen and just state that our model was wrong.]

  \pagenumbering{arabic}
  \section{Introduction and Background}
  	%% Main Aim
		

		%% Relevant materials * 3 - Some materials also on real world ant colonies
		% The below are a list of references that are not in the .bib file yet, will be added when I'm (Blayze) back home
		% 1) https://cs.gmu.edu/~sean/papers/panait04ant.pdf
		% 2) https://ieeexplore.ieee.org/document/4129846/ - ACO algorithm
		% 3) https://ieeexplore.ieee.org/document/4375636/ - ACO again
		% 4) https://link.springer.com/chapter/10.1007%2F978-3-540-28646-2_17 - Real ant simulation
		% 5) https://cs.gmu.edu/~sean/papers/panait04learning.pdf --- TODO: Read this some more, read the referenced papers, make sure it's being referenced properly

		A number of past studies have examined the way that ant colonies search their surrounding areas to locate and retrieve food \cite{vittori_modeling_2004, 1}.
		Often the aim of those studies focused on simulating ant colonies is to apply the emergent behaviour that ant colonies exhibit to solve abstracted technical
		problems \cite{2, 3}, with a smaller number of past works focusing on simulating ant colony behaviour as it is seen in nature \cite{4} %TODO: Perhaps cite some more things here
		
		\cite{4} paper implements ant trail following  ---- somehting about it modelling using deposition
		
		Whereas this \cite{1} introduces the concept of a dual-pheromone system, where one set of pheromone leads away, one toward
		
		\cite{5} discusses the benefits of simulating the ant's return path using pheromones or simply having the ants return directly

		TODO: Cite one other thing here, giving it another piece of information

		%% How the relevant materials inspired us
		These pieces of information inspired us to do stuff like the other stuff

  \section{Methodology}

  \section{Results}

  \section{Discussion}

  \section{Conclusions}

  % BIBLIOGRAPHY AND APPENDIX
  \newpage
  \bibliography{refs}
  \bibliographystyle{ieeetr}

  \newpage
  \begin{appendix}
    \listoffigures
    \listoftables
    %\listoflists
    \section*{Listings}
      %% We can put listings in here
  \end{appendix}
  
\end{document}
